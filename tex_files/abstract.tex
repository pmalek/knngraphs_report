This paper investigates assortativity of k-NN graphs constructed from real-world high-dimensional datasets.
Assortativity is the tendency of nodes to connect to other nodes that are similar to them.
After introducing the notion of k-NN graphs general problem definition is given regarding relation of $k$ parameter and graph's assortativity coefficient (also called Pearson correlation coefficient) in high dimensional datasets.
Highly dimensional datasets tend to contain hubs – instances whose in-degree (the number of incoming links attached to a node) is extremely high.
Large number of real-world network are correlated in the sense that the probability that a node of degree $d$ is connected to another node of degree $d'$ depends on $d$.
This paper investigates correlation of $k$ parameter of k-NN graphs to their assortativity index.
With positive assortative index $(0, 1>$, nodes tend to connect to other nodes that are similar.
With negative assortative index $<-1, 0$, nodes tend to connect to nodes that are different (have different degrees).
Values close to $0$ indicate no tendency of nodes to connect to any other particular nodes (random connections).
In this paper analysis of exemplar high-dimensional real-world datasets is given, basing on analyzing tool implemented for project's needs.
The discovered rules whether k-NN graphs representing real-world data are assortative, disassortative or show no degree correlations are given in the results section.