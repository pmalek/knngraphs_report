The basic idea of this paper is to investigate assortativity of k-NN graphs constructed from real-world high-dimensional datasets.
The k-NN graph is a graph in which two instances of the dataset P and Q are connected by a directed link if the distance between P and Q is among the k- th smallest distances from P to other instances.
In the case of highly dimensional data k-NN graphs contain so called hubs – instances whose in-degree (the number of in-coming links attached to a node) is extremely high.
A large number of real-world network are correlated in the sense that the probability that a node of degree d is connected to another node of degree $d'$ depends on $d$.
The aim of the study is to investigate properties of IDNN function for different k-NN graphs, where $IDNN(d)$ stands for the average in-degree of the nearest neighbors of k-NN nodes with in-degree $d$.
If there are no degree correlations then $IDNN(d)$ is independent of d, i.e. nodes connect to each other with the expected random probabilities.
If $IDNN(d)$ is an increasing function of d then the graph is assortative, i.e. hubs show a tendency to link to each other.
On the other hand if $IDNN(d)$ is a decreasing function of d then hubs tend to avoid linking between themselves (dissasortative mixing).
Degree correlations can be quantified by the slope of $IDNN(d)$ or by the Pearson correlation coefficient of degree between pairs of linked nodes (so called assortativity coefficient).

The project has to answer to the following research questions:
\begin{itemize}
\item Are k-NN graphs representing real-world, high dimensional datasets assortative, disassortative or show no degree correlations?
\item Is there any relation between degree correlations and k for different distance functions? In other words for different distance functions and different values of k what are the values of assortativity coefficient/the slope of IDNN function.
\end{itemize}
