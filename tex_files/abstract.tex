Most social, biological or technological networks which we describe as complex networks exhibit substantial nontrivial topological features, with some patterns of connections between their nodes that are not purely irregular.
Such features include amongst others assortativity or disassortativity of nodes.
Assortativity is the tendency of nodes to connect to other nodes that are similar to them.
Highly dimensional, real-world datasets tend to contain hubs – instances whose in-degree (the number of incoming links attached to a node) is extremely high.

This paper investigates investigates assortativity of k-NN graphs constructed from real-world high-dimensional datasets by means of trying to find correlation of $k$ parameter of k-NN graphs to graphs' assortativity index.
After introducing the notion of k-NN graphs general problem definition is given regarding relation of $k$ parameter and graph's assortativity coefficient (also called Pearson correlation coefficient) in high dimensional datasets.
With positive assortative index $(0, 1>$, nodes tend to connect to other nodes that are similar.
With negative assortative index $<-1, 0$, nodes tend to connect to nodes that are different (have different degrees).
Values close to $0$ indicate no tendency of nodes to connect to any other particular nodes (random connections).
In this paper analysis of exemplar high-dimensional real-world datasets is given, basing on analyzing tool implemented for project's needs.
The discovered rules whether k-NN graphs representing real-world data are assortative, disassortative or show no degree correlations are given in the results section.