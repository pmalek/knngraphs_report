\subsection{Used technologies}


%------------------------------------------
\subsubsection{JUNG library}
For creation and representation of graphs and coefficients calculations JUNG library~\cite{jung} has been used.
JUNG(Java Universal Network/Graph Framework) is a software library that provides a common and extendible language for modeling, analysis, and visualization of data that can be represented as a graph or network.

JUNG's \emph{DirectedSparseGraph} has been used as a class being the container for datasets' data.
Datasets' instances has been wrapped in custom \emph{Node} class, having a list of floating point values which were then used for distance calculations.


%------------------------------------------
\subsubsection{Distance algorithm}
For distance algorithm, indicating distance between \emph{Node}s, Euclidean distance algorithm has been implemented.
All values from \emph{Node}'s values list has been squared, summed and then square root of tham sum has been calculated.
The difference between values of nodes has been used to determine distances between \emph{Node}s.


%------------------------------------------
\subsubsection{Pearson correlation coefficient}
For Pearson correlation coefficient calculation a class from Apache Common Math library~\cite{apache_common_math} -\emph{org.apache.commons.math3.stat.correlation.PearsonsCorrelation} - has been used.

The implemented method can be seen in Listing~\ref{lst:pearsons_method}.

In this method for each link in the graph we take in-degrees from both of its ends - source node's in-degree and  destination node's in-degree - and then save it in arrays.
After the whole graph has been processed those arrays are then fed to \emph{PearsonsCorrelation} class and the coefficient is calculated with \emph{correlation} method.

\begin{filecode}[label=lst:pearsonCorrelation,caption=Method used for Pearson correlation coefficient calculation.]
  \label{lst:pearsons_method}
  \lstinputlisting{./code/pearsonCorrelation.java}
\end{filecode}

%------------------------------------------
\subsection{Implemented solution}
Implemented solution is a Java \emph{jar} file with following usage (replace \emph{VERSION} with appropriate version of the \emph{jar} file in possession).

\shellcmd{java -jar knngraphs-VERSION.jar -k K file1.csv [file2.csv ...]}

One can state one or more files (datasets) as shown above which will be used to create kNN graphs, show them and calculate Pearson coefficient vs $k$ graph for each file(dataset).
After launch, user will be asked until what parameter $k$ he wants the program to calculate the graphs if it has not been passed in as command line parameter.
After that the progm will start its calcaulations for all $k$s from 1 until $k$ given as parameter.
After that one can choose whether to show kNN graph for each $k$ ( Warning: it may be RAM and CPU heavy to show graphs for many $k$s) and whether to save Pearson coefficient vs $k$ graph to \emph{.svg} file in the working directory.

Implemented solution's source code has been push onto github repository~\footnote{Patryk Małek, kNN graphs, (2014), GitHub repository,~\url{https://github.com/pmalek/knngraphs}}.